Numerical simulation of physical phenomenal has long been of the interests in the area of mechanical engineering, physics, and in recent year computer graphics. Recent advancement of computation hardware has opened up new opportunities for improving the numerical simulations in terms of both scale and performance. But the heterogeneity of modern  hardware has imposed unique challenges that limit the utilization of computation hardware by using traditional coding paradigm. This dissertation takes the investigations of designing efficient Poisson and linear elasticity solvers for modern hardwares, to demonstrate some design practices and principles for utilizing modern hardwares in the context of numerical solvers.


%The research in the field usually falls into two categories: modeling and solver. Research in modeling intends in finding mathematical equations, sometimes with better accuracy, sometimes are easier to solve, that can be used to describe the physical phenomenal. Research in solvers, on the other hand, is to develop algorithms that are able to solve larger equations faster, for iterative approximate solvers, solve to higher accuracy.  This dissertation focus on designing efficient solvers, centering on a novel data structure namely SPGrid, which provides the flexibility of a sparse and adaptive data structure while retains the performance of a uniform grid. Furthermore, it will present the use of modern Same Instruction Multiple Data(SIMD) hardware and heterogenous platforms that exploits their high computation throughputs to saturate memory utility. Finally, the efficiency of those design principles and practices are demonstrated in the Poisson and Linear elasticity solvers.