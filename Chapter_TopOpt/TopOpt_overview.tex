\section{Method Overview}
Our sparse topology optimization framework consists of three key components: a sparse grid structure, a high-resolution multigrid FEM solver, and a narrow-band and unbounded structure optimizer. 
First, we briefly introduce the sparse paged structure (SPGrid) \cite{setaluri2014spgrid} as the base data structure to track the optimizing structures (Section \ref{sec:topopt_spgrid}). 
Next, we discuss our multigrid FEM solver for computing large-scale elastic systems on SPGrid in Section \ref{sec:topopt_multigrid}. 
Finally, in Section \ref{sec:topopt_result}, we provide method validations with respect to both the multigrid solver.

\section{Sparsely Populated Grid Structure}\label{sec:topopt_spgrid}
The Sparsely Populated Grid (SPGrid) data structure \cite{setaluri2014spgrid}, in comparison to other sparse  data structures, leverages the virtual memory system to allocate a very large virtual memory address span, corresponding to a sparsely populated background grid, while only materializing in physical memory the parts of this grid that are active. The allocation unit in SPGrid is a \emph{block}, a rectangular region of the Cartesian grid that is made contiguous in memory address space by virtue of a space-filling traversal scheme; The size of SPGrid blocks is chosen to be a multiple of a 4KB, i.e. the size of a physical memory page. SPGrid stores a number of data channels, which have similar sparsity pattern and corresponding indexing, in the same allocation block. Using the SPGrid data structure enables us to compute effectively on a sparsely populated domain with effective bandwidth comparable to a cache-optimized, dense uniform grid. In our multigrid FEM solver, we utilized the flexibility of SPGrid's block size, and chose a block size of $4\times 4\times 8$, tailored for vectorization as described in Section \ref{sec:topopt_multigrid}.